\documentclass[14pt,a4paper,final]{extreport}

\usepackage[utf8]{inputenc}
\usepackage[main=ukrainian,english]{babel}
\usepackage[left=2cm,top=2cm,right=1cm,bottom=2cm]{geometry}
\usepackage{graphicx}
\usepackage{titlesec}
\usepackage{verbatim}
\usepackage{float}
\usepackage{listings}
\usepackage{indentfirst}
\usepackage{minted}
\usepackage{hyperref}

\graphicspath{{img/}}
\setkeys{Gin}{width=\columnwidth}
\setminted{breaklines}
\lstset{breaklines}


% Some custom commands... 'cause I couldn't find the proper builtins 💀
\newcommand{\img}[2]{
% \vspace{\baselineskip}
\begin{figure}[H]
\includegraphics{#1}
#2
\end{figure}
}

% usage template
% \img{load_conf}{\caption{Завантаження конфігурації}}
% \section*{TODO}
% \addcontentsline{toc}{section}{TODO}

\title{Комп'ютерний практикум №1}
\date{}


\begin{document}
\begin{center}
    Міністерство освіти і науки України

    Національний технічний університет України

    "Київський політехнічний інститут імені Ігоря Сікорського"

    Фізико-технічний інститут
\end{center}
\vspace{\baselineskip}
\vspace{\baselineskip}
\vspace{\baselineskip}
\begin{center}
    Дисципліна "Теоретико-числовi алгоритми в криптологiї"
\end{center}
    \vspace{\baselineskip}
    \begin{center}
        Комп'ютерний практикум №3. Реалiзацiя та застосування алгоритму дискретного логарифмування index-calculus
    \end{center}
\vspace{\baselineskip}
\vspace{\baselineskip}
\vspace{\baselineskip}
\begin{center}
    Виконали
\end{center}
\begin{flushright}
    студенти гр. ФБ-11 Анучін М.К., Подолянко Т.О.
\end{flushright}
\vfill
\begin{center}
    Київ — 2024
\end{center}
\thispagestyle{empty}
\pagebreak

Мета роботи: ознайомитися з алгоритмом index-calculus пошуку дискретного логарифму, практично його реалізувати.
Дослідити переваги та недоліки даного алгоритму. Практично оцінити складність роботи алгоритму.

\section*{Постановка задачі}

Програмно реалізувати алгоритм пошуку дискретного логарифму index-calculus в двох модифікаціях: звичайний та з паралельним формуванням системи рівнянь.
Створити Docker image реалізованої програми. За допомогою задач, створених генератором, оцінити і порівняти ефективність реалізованих алгоритмів, 
створити візуалізацію.

\section*{Хід роботи}

Програмну релізацію виконано на Python з застосуванням numpy. Найбільші труднощі створила задача реалізувати алгоритм ровз'язання системи
лінійних рівнянь у кільці лишків за складеним модулем. Створена реалізація заснована на алгоритмі Гаусса. Рівняння які мають декілька розв'язків,
просто ігноруються, хоча і включаються у систему. Це є значним недоліком реалізації, оскільки витрачає процесорний час на роботу з рівняннями,
які не надають додаткової інформації для розв'язку.

Для знаходження канонічного розкладу застосовано програму, реалізовану у першій лабораторній роботі.

Для автоматизації збирання статистики про виконання було реалізовано функціонал взаємодії з генератором задач, автоматичного розв'язку та збереження 
статистик у форматі CSV (тип задачі, довжина $p$, $a$, $b$, $p$, розв'язок $x$, час пошуку у розв'язку у ns).

Реалізовано інтерфейс командного рядку для застосунку для демонстрації та тестування.

Вихідний код реалізованої програми розміщено на платформі GitHUb за посиланням \href{https://github.com/timofey282228/nta-lab3}{https://github.com/timofey282228/nta-lab3}.

Docker-контейнер з реалізованою програмою доступний за посиланням \url{https://hub.docker.com/repository/docker/timofey282228/nta-lab3}. 
Для встановлення Docker-контейнеру та роботи з програмою достатньо виконати такі команди:

\begin{minted}{bash}
docker pull timofey282228/nta-lab3
# Тепер образ контейнера доступний локально і з ним можна працювати
# Виведемо всі опції
docker run --rm -it timofey282228/nta-lab3 --help
\end{minted}

\section*{Реалізація розпаралелювання}

Оскільки використовувався Python, розпаралелювання за допомогою потоків дало б меншу перевагу, ніж за допомогою процесів, через GIL.
Тому для генерації рівнянь паралельно використано \verb|multiprocessing.Pool| з максимальною кількістю доступних процесів.
Але насправді значна частина процесорного часу витрачається на розв'язок системи лінійних рівнянь,
тому різниця між непаралельною та паралельною реалізаціями малі — вони розв'язують задачі однакового максимального порядку. 

\section*{Результати дослідження}

Задачі мають вид $x = \log_a{b}$. У випадку перевищення часу виконання у 5хв
програма завершується з повідомленням \verb|ERROR:WATCHDOG:Timed out|.

Приклад запуску:
\begin{lstlisting}
Solving tasks of length 3
Task type 1:
    a = 175; b = 130; p = 211.
    Solution: x = 169

Task type 2:
    a = 6; b = 143; p = 179.
    Solution: x = 91

Solving tasks of length 4
Task type 1:
    a = 2150; b = 115; p = 5717.
    Solution: x = 5167

Task type 2:
    a = 3932; b = 162; p = 5507.
    Solution: x = 2903

Solving tasks of length 5
Task type 1:
    a = 9770; b = 69694; p = 79549.
    Solution: x = 37363

Task type 2:
    a = 26999; b = 75793; p = 98837.
    Solution: x = 38212

Solving tasks of length 6
Task type 1:
    a = 465755; b = 25676; p = 631121.
ERROR:WATCHDOG:Timed out

\end{lstlisting}

\section*{Продуктивність роботи програмної реалізації}

\img{graph}

\section*{Оцінка максимального порядку вхідного параметра $p$}

Практично встановлено, що максимальний порядок параметра $p$, за якого задачу
дискретного логарифмування можна розв'язати за допомогою даної
реалізації алгоритму index-calculus за відведений час (5хв) (як паралельна, так і непаралельна реалізації), дорівнює 6.
Це гірше, ніж результат реалізації С-П-Г у попередній роботі. 

\section*{Висновки}

Алгоритм index-calculus дозволяє знайти дискретний логарифм
елемента мультиплікативної групи зі складністю $L_p [\frac{1}{2}, c]$.
Найефективніше цей алгоритм в класичній реалізації працює для груп простого порядку,
оскільки тоді розв'язання СЛР відбувається над полем, а не над кільцем.

Для ефективної імплементації із застосуванням розпаралелювання на етапі генерування
рівнянь складність інших етапів (зокрема, розв'язання системи) повинна бути порівняною.
В даній імплементації розв'язання системи є значно більш складною операцією.

\end{document}